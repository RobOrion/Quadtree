\documentclass[12pt,a4paper]{article}
\usepackage{charter}
\usepackage[french]{babel}
\usepackage[utf8]{inputenc}
\usepackage{lmodern}
\usepackage{color}
\usepackage{mathptmx}
\usepackage{fullpage}
\usepackage{graphicx}
\usepackage{amsmath}
\usepackage{amssymb}
\title{Projet : Quadtree}
\author{Hugo KANDEL, Robin CHATELET}
\date{Dimanche 8 janvier}
\begin{document}
\maketitle
\pagebreak
\newpage
\tableofcontents
\pagebreak
\newpage

\section{Organisation du programme}
\subsection{Problème posé}
L'objectif de ce projet et de réaliser dans le langage OCaml un programme permettant de manipuler des images, encodées sous forme d’arbre. Par soucis de simplification, on ne considère que les images carrées, donc de taille $2^n * 2^n$.
\subsection{Analyse du problème} 

\subsection{Avantages et inconvénients}
\subsubsection{Avantages}

\subsubsection{Inconvénients}

\section{Contribution au projet Quadtree}
Nous avons jugé le projet suffisamment court pour ne pas avoir à nous répartir les tâches de façon précises, via un tableau par exemple comme cela fut le cas pour le projet de fin d'année de CPI1. Nous avons avancés ensemble, à deux sur une même fonction ou séparément selon les besoins du moment.
\section{Bilan personnel}
\subsection{Robin CHATELET}

\subsection{Hugo KANDEL}
Ce projet fut intéressant dans la mesure où nous devions trouver une autre façon de résoudre un problème précédemment résolu.\newline
En effet, le "Projet Image" en CPI1 était sensiblement similaire : nous devions effectuer des actions sur une image au format ".ppm" afin de la modifier, puis de l'enregistrer.\newline
Ce fut une expérience enrichissante que d'aborder ce problème sous un autre jour, et surtout avec un autre langage.
\subsection{Remarques complémentaires}
PROJET QUI MARCHE PAS L.O.L
\end{document}
